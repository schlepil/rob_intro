\documentclass[usenames,dvipsnames,xcolor=table]{beamer}
\usepackage{subfiles}
\mode<presentation> {

% The Beamer class comes with a number of default slide themes
% which change the colors and layouts of slides. Below this is a list
% of all the themes, uncomment each in turn to see what they look like.

%\usetheme{default}
%\usetheme{AnnArbor}
%\usetheme{Antibes}
%\usetheme{Bergen}
%\usetheme{Berkeley}
\usetheme{Berlin}
%\usetheme{Boadilla}
%\usetheme{CambridgeUS}
%\usetheme{Copenhagen}
%\usetheme{Darmstadt}
%\usetheme{Dresden}
%\usetheme{Frankfurt}
%\usetheme{Goettingen}
%\usetheme{Hannover}
%\usetheme{Ilmenau}
%\usetheme{JuanLesPins}
%\usetheme{Luebeck}
%\usetheme{Madrid}
%\usetheme{CambridgeUS}
%\usetheme{Malmoe}
%\usetheme{Marburg}
%\usetheme{Montpellier}
%\usetheme{PaloAlto}
%\usetheme{Pittsburgh}
%\usetheme{Rochester}
%\usetheme{Singapore}
%\usetheme{Szeged}
%\usetheme{Warsaw}

% As well as themes, the Beamer class has a number of color themes
% for any slide theme. Uncomment each of these in turn to see how it
% changes the colors of your current slide theme.

%\usecolortheme{albatross}
%\usecolortheme{beaver}
%\usecolortheme{beetle}
%\usecolortheme{crane}
%\usecolortheme{dolphin}
%\usecolortheme{dove}
%\usecolortheme{fly}
%\usecolortheme{lily}
%\usecolortheme{orchid}
%\usecolortheme{rose}
%\usecolortheme{seagull}
%\usecolortheme{seahorse}
%\usecolortheme{whale}
%\usecolortheme{wolverine}

%\setbeamertemplate{footline} % To remove the footer line in all slides uncomment this line
%\setbeamertemplate{footline}[page number] % To replace the footer line in all slides with a simple slide count uncomment this line

%\setbeamertemplate{navigation symbols}{} % To remove the navigation symbols from the bottom of all slides uncomment this line
}

\usepackage[T1]{fontenc}
\usepackage[french]{babel}
\usepackage{csquotes}
\usepackage{verbatimbox}

\usepackage{graphicx}
\usepackage{amsmath}
\usepackage{amssymb}  % assumes amsmath package installed
\usepackage{amsfonts}
\usepackage{amsthm}

\DeclareSymbolFontAlphabet{\amsmathbb}{AMSb}%

\usepackage{nicefrac}
\usepackage{tikz}
\usepackage{sidecap}
%\IfFileExists{microtype.sty}{\usepackage{microtype}}{}

\usepackage{placeins}

\usefonttheme[onlymath]{serif}

\usepackage{relsize}
\usepackage{color}

\usepackage{newCommands}

\def\reff{\text{ref}}
\let\epsilon\varepsilon
\let\emptyset\varnothing

\usepackage{makecell}

%\usepackage{enumitem}
\usepackage{pifont}

\usepackage{lmodern}
%\usepackage{paralist}
\usepackage{enumerate}
\usepackage{varwidth} 
\usepackage{framed,color}
\definecolor{shadecolor}{rgb}{0.1, 0.6,0.1} 
\def\floatpagefraction{.8}

\beamertemplatenavigationsymbolsempty

\renewcommand{\circled}[1]{\tikz[baseline=(char.base)]{\node[shape=circle,draw,innersep=1pt] (char) {#1};}} 

%\newcommand{\results}[1]{{\small \textit{\textquote{#1}} }}
\newcommand{\results}[1]{{\small \hspace{20pt}\textit{- #1 -} }}
\newcommand{\loc}[1]{\hspace{20pt}{\small #1}}
\newcommand{\CC}{C\nolinebreak\hspace{-.05em}\raisebox{.4ex}{\tiny\bf+}\nolinebreak\hspace{-.10em}\raisebox{.4ex}{\tiny\bf +}}\def\CC{{C\nolinebreak[4]\hspace{-.05em}\raisebox{.4ex}{\tiny\bf ++}}}

\newcommand{\tblue}[1]{\textcolor{blue}{#1}}
\newcommand{\tfblue}[1]{\textcolor{blue}{\textbf{#1}}}

\date{\today} % Date, can be changed to a custom date

\graphicspath{{./figures/}}
\makeatletter
\def\input@path{{./figures/}}
\makeatother

\begin{document}
\begin{frame}
\begin{center}
    \textcolor{blue}{
    \textbf{
    {\Large
    Robotics from perspectives\\[4pt]
    }}
    }
    \rule{.9\linewidth}{2pt}\\[4pt]
    Philipp Schlehuber-Caissier \\
    Post-Doc, LRDE, Epita\\[6pt]
    \rule{.9\linewidth}{2pt}\\[4pt]
 	15 janvier 2019\\
    Orientation week
    
\end{center}
\end{frame}

%%%%%%%%%%%%%%%%%%%%%%%%%%%%%%%%%%%%%%%%%%%%%%%%%%%%%%%%

\section{Table of Contents}

\begin{frame}{Overview}
    \begin{itemize}
        \item The historical perspective
        \item The control theoretic perspective
        \item The industrial perspective
        \item My perspective
    \end{itemize}
\end{frame}

%%%%%%%%%%%%%%%%%%%%%%%%%%%%%%%%%%%%%%%%%%%%%%%%%%%%%%%%%%
\section{The historical perspective}


%%%%%%%%%%%%%%%%%%%%%%%%%%%%%%%%%%%%%%%%%%%%%%%%%%%%%%%%%%
\section{The control theoretic perspective}
\begin{frame}{Control theory - Intro}
A robot from a control theoretic perspective looks like this:
\begin{align*}
    \xd = f(\x) + g_{\uv}(\x, \uv) + g_{\omega_\x}(\x, \omega_\x)
    \y = h(\x) + g_{\omega_\y}(\omega_\y)
\end{align*}
\only<1->{
    \begin{columns}
    \begin{column}{0.5\linewidth}
        $\x$ : robot state\\
        $\y$ : ''observable´´ state
    \end{column}
    \begin{column}{0.5\linewidth}
        $f,g$ : system and input dynamics
        $\uv$ : control input
    \end{column}
    \end{columns}
}
\end{frame}
%%%%%%%%%%%%%%%%%%%%%%%%%%%%%
\begin{frame}{Control theory - Goal}
\only<1>{
Control theory seeks to generate a control signal $\uv(t)$ such that the state evolves in a desired way.
}
\only<2->{
\begin{center}
    {\large Centrifugal governor}\\
    Keeping steam engines at constant speed [James Watts 1788]
    \def\svgwidth{\linewidth}
    \input{reg.pdf_tex}
    \only<3>{
    \begin{minipage}{0.49\linewidth}
        opening angle depends on current speed
    \end{minipage}
    \begin{minipage}{0.49\linewidth}
        negative feedback between opening angle and 
    \end{minipage}
    }
\end{center}
}
\end{frame}
%%%%%%%%%%%%%%%%%%%%%%%%%%%%%%%%%%%%%%%%%%%%%%%%%%%%%%%%%%

\section{The industrial perspective}
\begin{frame}{The industrial perspective}
\begin{center}
    \Large{\textcolor{blue}{Three main applications}}
    \def\svgwidth{\linewidth}
    \input{industrial_robots1.pdf_tex}
\end{center}
\end{frame}

%%%%%%%%%%%%%%%%%%%%%%%%%%%%%
\begin{frame}{Robotic arm}
    \begin{center}
        \begin{minipage}{0.49\linewidth}
                \def\svgwidth{\linewidth}
                \input{robotic_arms.pdf_tex}
        \end{minipage}
        \hfill
        \begin{minipage}{0.49\linewidth}
        \large{Serial link robotic arm}
        \begin{itemize}
            \item[+] Large work-space
            \item[+] Versatility
            \item[]
            \item[-] Complexity due to redundancy
            \item[-] Relatively slow
        \end{itemize}
        \end{minipage}
    \end{center}
\end{frame}

\begin{frame}{Robotic arm - Frequent use case}
    \begin{center}
     % Video here
    \end{center}
\end{frame}
%%%%%%%%%%%%%%%%%%%%%%%%%%%%%
\begin{frame}{Parallel robot}
    \begin{center}
        \begin{minipage}{0.49\linewidth}
                \def\svgwidth{\linewidth}
                \input{pick_and_place.pdf_tex}
        \end{minipage}
        \hfill
        \begin{minipage}{0.49\linewidth}
        \large{Serial link robotic arm}
        \begin{itemize}
            \item[+] Speed & precision
            \item[+] Large ''work-space´´
            \item[]
            \item[-] Complexity due to internal constraints
            \item[-] Restricted orientation
        \end{itemize}
        \end{minipage}
    \end{center}
\end{frame}

\begin{frame}{Parallel robot - Frequent use case}
    \begin{center}
     % Video here
    \end{center}
\end{frame}
%%%%%%%%%%%%%%%%%%%%%%%%%%%%%
\begin{frame}{Mobile robots}
    \begin{center}
        \begin{minipage}{0.49\linewidth}
                \def\svgwidth{\linewidth}
                \input{logistics_1.pdf_tex}
        \end{minipage}
        \hfill
        \begin{minipage}{0.49\linewidth}
        \large{Serial link robotic arm}
        \begin{itemize}
            \item[+] Increases efficiency in warehouses
            \item[+] Semi-autonomous systems
            \item[]
            \item[-] Complexity due to multi-agent nature
            \item[-] Very task specific
        \end{itemize}
        \end{minipage}
    \end{center}
\end{frame}

\begin{frame}{Mobile robot - Use case}
    \begin{center}
     % Video here
    \end{center}
\end{frame}
%%%%%%%%%%%%%%%%%%%%%%%%%%%%%

%%%%%%%%%%%%%%%%%%%%%%%%%%%%%%%%%%%%%%%%%%%%%%%%%%%%%%%%

\end{document}

